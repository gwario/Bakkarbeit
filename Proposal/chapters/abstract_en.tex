\chapter*{Abstract}

\section{Problem definition}

In software engineering, an important part is to verify the functional correctness of a program. The difficulty of this task grows with the size and the complexity of a program. Thus the task of finding faulty code consumes a considerable amount of time and the efficiency of finding, understanding and fixing this faulty code is of major concern. A very efficient way to keep up code quality is to make developers understand how the code really works. In this thesis I'm going to implement a visualization of the program execution of Forth programs in Gforth and will analyse the improvement of user experience during debugging and time consumption of the process with example programs.

\section{Expected results}

Improvement of awareness of what's happening during the execution of a program and efficiency of finding faulty code.

\section{Methodology and approach}

As a first step I'm going to evaluate means of implementing a transparent way to generate a program trace by modification of the Gforth code. The next step is to visualize manipulation of the stack and accessed memory. Once a satisfying visualization in implemented, I'm going to debug example programs with an without the visualization to verify my assumption.

\section{State of the art}

Current methods of locating faults are

\subsection{Print debugging}

Words like \emph{.} \texttt{.}, \texttt{.\textquotedbl} and \texttt{\textasciitilde\textasciitilde} print information directly to the terminal.

\subsection{Gforth debugger}

Stepping throw program execution with dbg.

\subsection{Writing test cases}

Writing test cases for words to narrow down the actual location of the fault.

\section{Relation to Software engineering}

\begin{itemize}
	\item Software quality assurance (testing, dynamic analysis, debugging)
	\item Software development methodology (prototyping, agile)
	\item Stack-based language(forth)
\end{itemize}

\section{Timetable}

\begin{enumerate}
  \item research on program execution/trace visualization
	\item research on forth
  \item research on architecture of gforth
  \item research on ``debugging'' in forth/gforth
  \item research of similar approaches
  \item extracting several technical approaches to accomplish the task(hooks, word-wrapping, level of implementation)
  \item evaluation of the approaches(automation?, performance, feasibility)
  \item prototyping the approaches in order of quality
  \item evaluation visualization methods
  \item implementation of one method
\end{enumerate}


\section{References}

\begin{itemize}
	\item print debugging
	\item visualization of program execution/traces
	\item debugger
\end{itemize}