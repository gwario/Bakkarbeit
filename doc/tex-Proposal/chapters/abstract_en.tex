\chapter*{Abstract}

\section{Problem definition}

In software engineering, an important part is to verify the functional correctness of a program. The difficulty of this task grows with the size and the complexity of a program. Thus the task of finding faulty code consumes a considerable amount of time and the efficiency of finding, understanding and fixing this faulty code is of major concern. A very efficient way to keep up code quality is to make developers understand how the code really works. In this thesis I'm going to implement a visualization of the program execution of Forth programs in Gforth and will analyse the improvement of user experience during debugging and time consumption of the process with example programs.

\section{Expected results}

Improvement of awareness of what's happening during the execution of a program and efficiency of finding faulty code.

\section{Methodology and approach}

As a first step I'm going to evaluate means of implementing a transparent way to generate a program trace by modification of the Gforth code. The next step is to visualize manipulation of the stack and accessed memory. Once a satisfying visualization in implemented, I'm going to write and debug example programs with an without the visualization to verify my assumption.

\section{State of the art}

Current methods of locating faults are

\subsection{Print debugging}

Words like \emph{.} \texttt{.}, \texttt{.\textquotedbl} and \texttt{\textasciitilde\textasciitilde} print information directly to the terminal.

\subsection{Gforth debugger}

Stepping throw program execution with dbg.

\subsection{Writing test cases}

Writing test cases for words to narrow down the actual location of the fault.

\section{Relation to Software engineering}

\begin{itemize}
	\item Software quality assurance (testing, dynamic analysis, debugging)
	\item Software development methodology (prototyping, agile)
	\item Stack-based language(forth)
\end{itemize}

\section{Timetable}

\begin{center}
    \begin{tabular}{ | c | p{11cm} |}
    \hline
    Calendar week & work \\ \hline
    2014 - 40 & Research on forth, the architecture of gforth, ``debugging'' in forth/gforth,  program execution/trace visualization and on similar approaches \\ \hline
    2014 - 44 & Extracting several technical approaches to accomplish the task(hooks, word-wrapping, level of implementation, ...) \\ \hline
    2014 - 45 & Evaluation of the approaches(automation, performance, feasibility) \\ \hline
    2014 - 46 & Prototyping the approaches in order of quality \\ \hline
	  2014 - 50 & Evaluation visualization methods \\ \hline
    2014 - 52 & Verification of the hypothesis \\ \hline
    2015 - 03 & Final documentation and feedback cycles \\ \hline
    2015 - 05 & Submission \\
    \hline
    \end{tabular}
\end{center}


\section{References TODO}

\begin{itemize}
	\item print debugging
	\item test driven development
	\item visualization of program execution/traces
	\item debugger
\end{itemize}