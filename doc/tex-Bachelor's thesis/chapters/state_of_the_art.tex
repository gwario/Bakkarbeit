%% intro.tex
%% Copyright (C) 2014 by Thomas Auzinger <thomas.auzinger@cg.tuwien.ac.at>
%
% This work may be distributed and/or modified under the
% conditions of the LaTeX Project Public License, either version 1.3
% of this license or (at your option) any later version.
% The latest version of this license is in
%   http://www.latex-project.org/lppl.txt
% and version 1.3 or later is part of all distributions of LaTeX
% version 2005/12/01 or later.
%
% This work has the LPPL maintenance status `maintained'.
%
% The Current Maintainer of this work is Thomas Auzinger.
%
% This work consists of the files vutinfth.dtx and vutinfth.ins
% and the derived file vutinfth.cls.
% This work also consists of the file intro.tex.


\newacronym{ctan}{CTAN}{Comprehensive TeX Archive Network}
\newacronym{faq}{FAQ}{Frequently Asked Questions}
\newacronym{pdf}{PDF}{Portable Document Format}
\newacronym{svn}{SVN}{Subversion}
\newacronym{wysiwyg}{WYSIWYG}{What You See Is What You Get}

\newglossaryentry{texteditor}
{
  name={editor},
  description={A text editor is a type of program used for editing plain text files.}
}

\chapter{State of the art / analysis of existing approaches}

\section{literature studies}

\begin{itemize}
\item about program comprehension
	\begin{itemize} \item factoring \item displayed depth of factoring \end{itemize}
	\begin{itemize} \item about debugging \end{itemize}
	\begin{itemize}
	\item dataflow analysis(Backward Analysis)(not sufficient in demo) \\
		Darren C. Atkinson , William G. Griswold, Implementation Techniques for Efficient Data-Flow Analysis of Large Programs, Proceedings of the IEEE International Conference on Software Maintenance (ICSM'01), p.52, November 07-09, 2001
	\end{itemize}
\item about debugging in other paradigms
	\begin{itemize} \item (?)about some tools \end{itemize}
\item about debugging in stack oriented languages
	\begin{itemize} \item (?)about some tools \end{itemize}
\item (?)about visualization maybe some examples and tools
	\begin{itemize}
	\item sequence diagram
	\item interaction diagrams (Jacobson, 1992)/ scenario diagrams (Koskimies and Mössenböck 1996)
	\item information murals (Jerding and Stasko, 1998)
	\item polymetric views (Ducasse et al., 2004)
	\item hierarchical edge bundling (Holten, 2006)
	\item structural and behavioral views of object-oriented program (Kleyn and Gingrich, 1988)
	\item matrix visualization and “execution pattern” notations \cite{Pauw98executionpatterns} to visualize traces in a scalable manner(De Pauw et al. 1993, 1994, 1998) 
	\item architecture oriented visualization (Sefika et al. 1996)
	\item a continuous sequence diagram, and the “information mural” (Jerding and Stasko, 1998)
	\item architecture with dynamic information (Walker et al. 1998)
	\item frequency spectrum analysis (Ball 1999)
	\end{itemize}
\item (?)about realtime/interactive vs post mortem
\end{itemize}

\section{analysis}

\begin{itemize}
\item existing methods abstract(abstract like print debugging and stepping and so on)
\item applicability for so-languages
\end{itemize}

\section{comparison and summary of existing approaches}

\begin{itemize}
\item existing methods(actual methods)
	\begin{itemize}
	\item factoring (http://en.wikipedia.org/wiki/Modular\_programming https://www.complang.tuwien.ac.at/forth/gforth/Docs-html/Factoring-Tutorial.html http://www.ultratechnology.com/Forth-factors.htm)
	\item dump
	\item . / type
	\item dbg
	\item see/ code-see
	\item \textasciitilde\textasciitilde
	\end{itemize}
\end{itemize}