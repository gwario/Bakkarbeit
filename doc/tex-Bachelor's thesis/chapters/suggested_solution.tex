%% intro.tex
%% Copyright (C) 2014 by Thomas Auzinger <thomas.auzinger@cg.tuwien.ac.at>
%
% This work may be distributed and/or modified under the
% conditions of the LaTeX Project Public License, either version 1.3
% of this license or (at your option) any later version.
% The latest version of this license is in
%   http://www.latex-project.org/lppl.txt
% and version 1.3 or later is part of all distributions of LaTeX
% version 2005/12/01 or later.
%
% This work has the LPPL maintenance status `maintained'.
%
% The Current Maintainer of this work is Thomas Auzinger.
%
% This work consists of the files vutinfth.dtx and vutinfth.ins
% and the derived file vutinfth.cls.
% This work also consists of the file intro.tex.


\newacronym{ctan}{CTAN}{Comprehensive TeX Archive Network}
\newacronym{faq}{FAQ}{Frequently Asked Questions}
\newacronym{pdf}{PDF}{Portable Document Format}
\newacronym{svn}{SVN}{Subversion}
\newacronym{wysiwyg}{WYSIWYG}{What You See Is What You Get}

\newglossaryentry{texteditor}
{
  name={editor},
  description={A text editor is a type of program used for editing plain text files.}
}

\chapter{Suggested solution/implementation}

\begin{itemize}
\item kind of an ide
\begin{itemize}
\item interactive program manipulation: state of the system before a word, after a word and by clicking on the word jumping to its definition and there also providing those features
\item stepping debugger mode: simply stepping through the whole code word by word
\item other data structures and variables should be displayed
\item display of the 'vocabulary'
\end{itemize}
\item proof of concept by enhancement of stepping debugger on forth code level(cause it has turned out to be the fastest and simples approach) by showing additional data: the other stacks
\end{itemize}