%% intro.tex
%% Copyright (C) 2014 by Thomas Auzinger <thomas.auzinger@cg.tuwien.ac.at>
%
% This work may be distributed and/or modified under the
% conditions of the LaTeX Project Public License, either version 1.3
% of this license or (at your option) any later version.
% The latest version of this license is in
%   http://www.latex-project.org/lppl.txt
% and version 1.3 or later is part of all distributions of LaTeX
% version 2005/12/01 or later.
%
% This work has the LPPL maintenance status `maintained'.
%
% The Current Maintainer of this work is Thomas Auzinger.
%
% This work consists of the files vutinfth.dtx and vutinfth.ins
% and the derived file vutinfth.cls.
% This work also consists of the file intro.tex.


\newacronym{ctan}{CTAN}{Comprehensive TeX Archive Network}
\newacronym{faq}{FAQ}{Frequently Asked Questions}
\newacronym{pdf}{PDF}{Portable Document Format}
\newacronym{svn}{SVN}{Subversion}
\newacronym{wysiwyg}{WYSIWYG}{What You See Is What You Get}

\newglossaryentry{texteditor}
{
  name={editor},
  description={A text editor is a type of program used for editing plain text files.}
}

\chapter{Introduction}

\section{Motivation}

\begin{itemize}
\item industrial software -> bug -> statistics -> understand(up to 60\% \cite{Pigoski:1996:PSM:524398}) to fix
	\begin{itemize}
	\item Industrial software, due to its(steady growning) complexity \cite{Lehman:1985:PEP:7261}(need to read) \\ structured programming http://dl.acm.org/citation.cfm?id=1243380
	\item The mental model of the client -> the specification(formal and informal) -> the mental model of developement team and the programmers -> the actual written behavior.
	\item software evolution \\
		Evelyn Barry , Sandra Slaughter , Chris F. Kemerer, An empirical analysis of software evolution profiles and outcomes, Proceedings of the 20th international conference on Information Systems, p.453-458, December 12-15, 1999, Charlotte, North Carolina, USA
	\item maintainance \cite{Lientz:1980:SMM:601062} \cite{ISOSWMaintainance} \\
		T. H. Ng , S. C. Cheung , W. K. Chan , Y. T. Yu, Do Maintainers Utilize Deployed Design Patterns Effectively?, Proceedings of the 29th international conference on Software Engineering, p.168-177, May 20-26, 2007
	\item In all these transitions information maybe lost and thus some elements of the chain are encouraged to change behavior.
	\end{itemize}
\item code gets written one time and read 4 times
\item program comprehension
	\begin{itemize}
	\item strategies \cite{Storey:1999:CDE:308936.308940}
		\begin{itemize}
		\item top down
		\item bottom up
		\item knowledgebased
		\item systematic and as-needed
		\item integrated approaches
		\end{itemize}
	\item dynamic analysis as defined by \cite{Ball:1999:CDA:318774.318944} \cite{Cornelissen:2009:SSP:1638616.1639301}
		\begin{itemize}
		\item actual behavior
		\item incomplete view \cite{Ball:1999:CDA:318774.318944}
		\item observer effect \\
		Andrews, J. (1997). Testing using log file analysis: tools, methods, and issues.
		In Proc. International Conference on Automated Software Engineering (ASE), pages 157–
		166. IEEE Computer Society Press
		\item scalability \\
		Zaidman, A. (2006). Scalability Solutions for Program Comprehension through Dynamic
		Analysis. PhD thesis, University of Antwerp
		\end{itemize}
	\item static analysis as defined by \cite{Ball:1999:CDA:318774.318944}
		\begin{itemize}
		\item ...
		\end{itemize}
	\item mental model
	\item documentation
	\item source code level documentation \\ Ninus Khamis , Juergen Rilling , René Witte, Assessing the quality factors found in in-line documentation written in natural language: The JavadocMiner, Data \& Knowledge Engineering, 87, p.19-40, September, 2013
	\item requirements for tools
	\end{itemize}
\item debugging -> different kind of paradigms and languages and tools
\item concatenative languages -> forth, postscript, factor
\item comparisson to oo langs
\end{itemize}

David Shepherd , Lori Pollock , K. Vijay-Shanker, Case study: supplementing program analysis with natural language analysis to improve a reverse engineering task, Proceedings of the 7th ACM SIGPLAN-SIGSOFT workshop on Program analysis for software tools and engineering, p.49-54, June 13-14, 2007, San Diego, California, USA \\

Martin P. Robillard , Wesley Coelho , Gail C. Murphy, How Effective Developers Investigate Source Code: An Exploratory Study, IEEE Transactions on Software Engineering, v.30 n.12, p.889-903, December 2004\\

Darren C. Atkinson , William G. Griswold, The design of whole-program analysis tools, Proceedings of the 18th international conference on Software engineering, p.16-27, March 25-29, 1996, Berlin, Germany\\

\section{problem statement (which problem should be solved?)}

\begin{itemize}
\item much work and tools on oo-languages
\item not so much on concatenative \sout{stack oriented} languages
\item applicability of oo-methods for concatenative \sout{stack oriented} languages at the example of forth
\item applicability of oo-visualization methods
\end{itemize}

\section{aim of the work}

\begin{itemize}
\item identify important information
\item visualization of information
\item demo approach
\end{itemize}

\section{methodological approach}

\begin{itemize}
\item qualitative approach(?)
\item proposal
\item Preliminary evaluations as defined by \cite{Cornelissen:2009:SSP:1638616.1639301}
\item outcome is a subjectiv view of the available methods, and proposed enhancements which have been implemented and suggestions of further enhancements
\end{itemize}

\section{structure of the work}

\begin{itemize}
\item summary on the available methods for program comprehension in gforth
\item summary and applicability of available methods for other paradigms and languages
\item enhancement of existing methods and proposal for further enhancements
\item poc
\end{itemize}