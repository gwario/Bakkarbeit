\chapter{Introduction}

\section*{Motivation}

Software systems underlie continuous changes throughout their whole life cycle.
They evolve from the beginning of development until their release and maintenance phase. Large software systems\footnote{"The term large is, generally, used to describe software whose size in number of lines of code is greater than some arbitrary value. For reasons indicated in [leh79], it is more appropriate to define a large program as one developed by processes involving groups with two or more management levels."\cite{Lehman:2003:SEB:950401.950407}} and most of all, software  classified as type E \cite{Cook:2006:ESS:1115566.1115567}, get more complex over time.\\
If there are more than a few developers/development teams involved or the developers/development teams are spread allover the world, there exists more foreign code than self written code. This makes it necessary to understand foreign code.
Most of all when it comes to making changes, enhancements or fixes, there is a very high level of understanding of the software at hand necessary\cite{Boehm:1976:SE:1311958.1312684}\cite{Singer97anexamination}. Due to \cite{Cornelissen:2009:SSP:1638616.1639301}, "... up to 60\% of the maintenance effort is spent on gaining a sufficient understanding of the program ...". These facts emphasize the need for computer aided methods to improve program understanding.

This thesis addresses the task of improving program comprehension of the concatenative programming language Forth on several levels. The terms "program understanding" and "program comprehension" are considered synonym. Namely the reading of source code, static analysis, dynamic analysis and the assistance of writing readable and easy to understand source code.

Due to the nature of concatenative languages, it is possible to write source code which reads very similar to natural language. There are no hard boundaries to the structure of the source code(custom defined loops and control structures) as in most other languages. Since Forth directly operates only on stacks and memory, the information which is immediately needed to follow the program execution is limited to those structures. In contrast, in object oriented languages there is also object state, object life cycle and concurrency to keep in mind.

\begin{comment}
\hl{mental model(LaToza et al., 2006) read: ..comment}
\@inproceedings\{Lieberman:1995:BGC:223904.223969,
author = \{Lieberman, Henry and Fry, Christopher\},
title = \{Bridging the Gulf Between Code and Behavior in Programming\},
booktitle = \{Proceedings of the SIGCHI Conference on Human Factors in Computing Systems\},
series = \{CHI '95\},
year = \{1995\},
isbn = \{0-201-84705-1\},
location = \{Denver, Colorado, USA\},
pages = \{480--486\},
numpages = \{7\},
url = \{http://dx.doi.org/10.1145/223904.223969\},
doi = \{10.1145/223904.223969\},
acmid = \{223969\},
publisher = \{ACM Press/Addison-Wesley Publishing Co.\},
address = \{New York, NY, USA\},
\}
\end{comment}


\section*{Problem statement}

There has been done plenty of work on the task of program comprehension in object oriented and procedural languages\cite{Cornelissen:2009:SSP:1638616.1639301}, but nearly none for concatenative languages. \begin{comment} The qualitative exploratory approach of this thesis does not encourage the formulation of specific hypothesis. Therefore the first question, is the applicability of existing methods and their visualization techniques. The second question to be answered, concerns new approaches, which may be exclusive to concatenative languages or Gforth/Forth.\end{comment}

\section*{Aim of the work}

This work aims to give a brief overview on the field of program comprehension and its methods, to show existing aids to program understanding in Gforth/Forth and to study the applicability of some existing analysis and visualization approaches for other paradigms. Further more the suggestion of new methods or the modification of existing methods to meet the characteristics of Gforth/Forth.

\section*{Structure of the work}

In Chapter \ref{chap:Methodology}, I will line out the methodological approach of this thesis. In Chapter \ref{chap:StateOfTheArt}, there will be a brief overview on the topic of program comprehension, an overview on the existing tools to improve program understand in general and on those specific to Gforth/Forth and afterwards I will present a selection of visualization approaches for procedural and object oriented languages. In Chapter \ref{chap:Results}, I will present some of the previously mentioned approaches for other paradigms applied on a real world Forth program, analyze their applicability and propose modification to those approaches. Afterwards I will present a prototype implementation of a program trace visualization enhancement to Gforth. Chapter \ref{chap:Summary} concludes the thesis with a summary and topics of further investigation.

\begin{comment}
At first, the available information of a forth program is identified. The next step is to characterize the information and its necessity for program comprehension is investigated. The differences of forth and object oriented languages are summarized and then the applicability of existing analysis and visualization methods is presented. \hl{Since there is no standard implementation of object orientation if forth, this thesis won't take any object orientation implementation into account.}
The last part of this thesis investigates probable enhancements and modifications to existing methods and proposes new approaches.
After the conclusion, the thesis presents further suggestions to support program comprehension and further topics of research in this direction.
\end{comment}